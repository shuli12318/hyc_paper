%1. 规范硕士导言
% \documentclass[master,ttf]{nudtpaper}

%2. 规范博士导言
% \documentclass[doctor,twoside,ttf]{nudtpaper}

%3. 建议使用OTF字体获得较好的页面显示效果
%   OTF字体从网上获得,各个系统名称统一。

%4. 如果想生成盲评,传递anon即可,仍需修改个人成果部分
% \documentclass[master,otf,anon]{nudtpaper}

\documentclass[master,ttf,twoside]{nudtpaper}
\usepackage{mynudt}
\usepackage{multirow,array,makecell}


\usepackage{graphicx}  % 用于插入图像
\usepackage{subfig}     % 用于子图
% 或者使用 subcaption 包
%\usepackage{subcaption} % 另一个可选的子图包



\usepackage{algorithm}
\usepackage{algorithmic}


%中文封面
\classification{TP242.6}
\serialno{21053048}
\confidentiality{公开}
\UDC{681.5}
\papertype{工学}
\title{面向地铁排爆环境的\\救援机器人RGB-D语义分割方法研究}
\displaytitle{面向地铁排爆环境的救援机器人RGB-D语义分割方法研究} %
\author{郝以慈}
\subject{控制科学与工程}
\researchfield{智能机器人技术}
\supervisor{周宗潭\quad{}教授}
\cosupervisor{XXX\quad{}讲师}  % 协助指导教师,没有就空着
\zhdate{\zhtoday}

%英文封面
\entitle{RGB-D Semantic Segmentation for Rescue Robots in Explosive Ordnance Disposal Environments}
\enauthor{Yici Hao}
\ensupervisor{Prof. Zongtan Zhou}
\encosupervisor{Lec. Jingsheng Tang} % 协助指导教师英文,没有就空着
\ensubject{Control Science and Engineering}
\enpapertype{Engineering}
\endate{\entoday}
% 加入makenomenclature命令可用nomencl制作符号列表。









% 论文主体
\begin{document}
\graphicspath{{figures/}} % 图片路径
\maketitle       % 生产封面

% 前言
\frontmatter
\tableofcontents % 生成文目录
\listoftables    % 生成表目录
\listoffigures   % 生成图目录

% 摘要
\midmatter
\input{data/abstract}   % 生成摘要
\input{data/denotation} % 生产符号目录

% 正文计数设置
\mainmatter
\renewcommand\UrlFont{\timesnr} % 定义URL字体
\makeatletter                   % 使用@符号,跟自动计数有关,不需要管
\newcounter{blankpages}         % 页面计数器
\def\cleardoublepage{           % 保证章节从奇数页开始
	\clearpage
	\if@twoside
	\ifodd\c@page
	\else
	\hbox{}\newpage\stepcounter{blankpages}%
	\thispagestyle{empty}%
	\if@twocolumn\hbox{}\newpage\fi
	\fi
	\fi
}
\newcommand{\@romannoblank}[1]{  % 用于生成不包含空白页的罗马数字
	\@roman{\numexpr#1-\value{blankpages}\relax}%
}
\makeatother                     % 使用@符号,跟自动计数有关,不需要管

% 正文
\chapter{绪论}

\iffalse           %这是注释
\begin{figure}[htp]
	\centering
	\includegraphics{picmain}
	\caption{图 1.1 名称}
\end{figure}

\begin{table}[htp]
	\centering
	\begin{minipage}[t]{0.8\linewidth} % 如果想在表格中使用脚注,minipage是个不错的办法
		\caption[表 1.1 名称]{}
		\begin{tabular*}{\textwidth}{lp{10cm}}
			\toprule[1.5pt]
			{\hei 列1} & {\hei 列2} \\
			\midrule[1pt]
			&  \\
			& \\
			& \\
			& \\
			& \\
			& \\
			\bottomrule[1.5pt]
		\end{tabular*}
	\end{minipage}
\end{table}


\cite{}

\fi



\section{研究背景与意义}


\subsection{研究背景}
地铁能够在短时间内运送大量乘客,从而减少城市拥堵、缓解地面交通压力,是现代城市交通的重要组成部分。
但是,受到公司运营成本、人员流动量大和人员流动速度较快等因素的限制,我国很多城市的地铁安检工作却较为宽松,并不能有效地防止爆炸物进入地铁环境。
爆炸事件一旦发生,将对地下丛横交错的城市水网、城市电网等公共设施造成破坏,
不仅会对疏散乘客和修复基础设施等工作带来诸多不便,危害公众的生命财产安全,还会对城市的高效运转和社会经济的快速发展产生巨大的不利影响。

传统的地铁排爆工作主要依赖排爆专家等技术人员深入爆炸环境排除爆炸物。然而,传统排爆方法却有很多弊端。
首先,爆炸物危害极大,一旦发生爆炸,将对排爆专家生命安全造成巨大危害。
其次,地铁内部的排椅和扶杆等结构化物体挤占了地铁内部空间,导致地铁内部空间较为狭窄。如果排爆专家身着排爆服进行排爆作业,空间移动将会收到很大限制。

随着智能机器人技术的快速发展,救援机器人在排爆领域逐渐开展应用。相比于传统排爆方法,救援机器人排爆展现出很多优势。
首先,救援机器人具有一定的自主能力,可以代替人类进入危险区域进行排爆作业,极大地降低了人员伤亡的风险。
其次,救援机器人体型较小便于在地铁排爆环境中移动,视角较低便于发现座椅下等隐匿的爆炸物,方便快速推进排爆工作。
再次,救援机器人可以携带多种传感器和设备,提高排爆的效率和准确性。

推动救援机器人在排爆环境中的自主作业,需要救援机器人感知周边环境。常见的传感器有激光雷达、RGB相机和RGB-D相机。
其中,激光雷达通过发射激光并捕捉物体表面反射的激光来生成高精度的三维点云数据。
但是,激光雷达一方面存在近距离盲区导致无法感知距离较近的物体,在狭窄的地铁环境中不利于感知环境。
另一方面,激光雷达获取环境数据稀疏,并且没有颜色信息和纹理信息。此外,激光雷达的成本也比较高。
RGB相机可以获取场景中物体丰富的颜色信息、形状信息和纹理信息,但是无法获取深度信息。
RGB-D相机集成了RGB相机和深度相机的功能。其中,RGB相机捕捉彩色图像,深度相机负责测量每个像素点到相机的距离,生成深度图像。
因此,RGB-D相机能够提供丰富的视觉信息,助力在狭窄的地铁环境中感知环境。

语义分割技术是地铁排爆场景语义理解的核心技术之一。相比于基于边框的目标检测而言,语义分割能够对场景进行更加精细和准确的解析。
语义分割通过通过给图像中的每个像素分配一个特定的类别,实现像素级图像理解。
通过像素级的图像理解,语义分割可以理解场景结构和布局,精确地识别和分类图像中的不同对象,为救援机器人在地铁排爆场景中的定位、建图、规划和决策等任务提供丰富的语义信息。

RGB-D语义分割的数据由彩色图像和对应的深度图像组合而成,两种图像虽然具有相同的数据结构,但是模态不同、数据内容不同,属于异质数据。
如何获取不同模态数据的有效信息、利用不同模态数据的互补性来提高语义分割的准确度是RGB-D语义分割的核心问题。
为了提升救援机器人感知系统的能力,本文的研究聚焦于RGB-D语义分割的异质数据特征提取与融合这个关键内容。


\subsection{研究意义}
本文的研究意义在于以下XX个方面:
(1)本文构建了一个地铁排爆场景的语义分割数据集,包括地铁闸机、楼梯、地铁内部常见物体以及模拟的管状爆炸物,覆盖了地铁场景从进站到乘车的几种常见场景。
(2)本文提出了一种XX方法。
(3)本文提出了一种XX方法。
(4)本文将上述两种算法应用到实际的救援机器人上,能够使救援机器人更好地理解地铁排爆场景,为救援工作的定位、建图、规划和决策提供丰富语义信息。
由此可见,本文的研究能为救援机器人提供可靠的语义信息,理解地铁排爆场景,进一步推动救援机器人全自主地铁排爆,具有较强的实际应用价值。


\section{国内外研究现状}

\subsection{三维传感器的研究现状}
相比于二维传感器而言,三维传感器可以捕获深度,获得场景的空间信息。常见的三维传感器主要包括激光雷达和RGB-D相机。
如\ref{图:常见的三维传感器} 所示。
\begin{figure}[h]
	\centering%
	\subfloat[激光雷达]{%
		\label{fig:rescue_1}
		\includegraphics[width=0.15\textwidth]{figures/激光雷达.png}}
	\subfloat[双目深度相机]{%
		\label{fig:rescue_3}
		\includegraphics[width=0.25\textwidth]{figures/双目深度相机.png}}
	\subfloat[TOF深度相机]{%
		\label{fig:rescue_2}
		\includegraphics[width=0.25\textwidth]{figures/TOF深度相机.png}}
	\subfloat[结构光深度相机]{%
		\label{fig:rescue_4}
		\includegraphics[width=0.25\textwidth]{figures/结构光深度相机.png}}
	\vspace{-1em}
	\caption{常见的三维传感器}
	\label{图:常见的三维传感器}
\end{figure}

在激光雷达中,由激光器发射出的脉冲激光在打到周围物体的表面时会发生反射,一部分反射激光会被激光雷达的接收器捕获,通过分析激光遇到目标对象后的在空中的折返时间,可以计算出激光雷达到目标对象的距离。
激光雷达通过从上到下逐层发射脉冲激光,得到目标对象上全部目标点的数据,比如空间坐标、反射率和表面纹理等,通过对这些数据进行成像处理,得到精准的三维点云图像。
根据成像原理,激光雷达有效探测距离可达上百米。但是激光雷达存在近距离盲区,无法捕捉自身周围1米内的物体信息。因此,激光雷达更适用于室外场景。


RGB-D相机整合了RGB相机和深度相机的优势,将两者集于一身。在获得彩色图像的同时,也获得了对应的深度信息,继而实现对周围环境的三维信息捕获。
深度相机的探测距离较短,但是近距离视野盲区半径可低至30厘米左右,相比于激光雷达,更适合在室内场景作业的机器人进行环境感知。


根据获取三维数据时是否主动发射出光波,RGB-D相机主要分为三种:被动式、主动式和多模态融合式。
依据这三种分类,发展出了不同的商用方案,主要有:双目视觉方案、飞行时间(Time of Flight, TOF)方案、结构光方案。

在双目视觉方案中,相机有两个类似人眼位置排列的RGB相机,利用这两个RGB相机获得不同位置得到的图像数据,基于图像特征点匹配原理对有视角差异的图像数据进行映射,通过计算图像特征点间的位置偏差,可以获取物体三维几何信息。
双目相机主要优点在于其硬件要求低,因此成本也低。
但是缺点也很明显。一方面,由于纯视觉方案对图像进行特征点的提取和匹配计算量大,因此对算法要求高,实时性效果较差。
另一方面,因为只有RGB一种传感器,对环境特征点的提取和匹配误差会较大,因此算法产生的深度精度不高。
如果场景特征不明显还会导致匹配失败,因此对纹理单调的场景不适用。
目前,双目深度相机在国外比较知名的生产商有:大疆,Intel, Stereolabs,Leap Motion等。


在飞行时间方案中,相机通过连续主动发射激光脉冲,再用传感器接收返回的激光脉冲,利用激光脉冲在空中的飞行时间来计算相机到目标物体的深度信息。
由于使用激光脉冲进行深度特测,因此TOF深度相机检测距离远,在激光能量充足的条件下可达几十米。
此外,激光脉冲在受环境光干扰的情况下,对成像效果影响较小。
但是,TOF深度相机的成本相对较高、体积较大。
目前,TOF深度相机在国外比较知名的生产商有:海康威视、联想、MicroSoft等。


在结构光方案中,相机通过近红外发射器,将具有一定结构特征的红外光线连续投影到目标三维空间的表面,然后基于接收模组收集并解析被物体表面反射回来的红外光线,得到物体的位置和深度信息。
因为,这种具备一定结构特征的红外光线会随着被摄物体深度的不同而产生不同的相位信息,运算单元可以将这种结构特征的变化换算成深度信息,并以此来获得三维结构。
结构光深度相机在近距离范围内精度和分辨率较高高,帧率可达60FPS。
但是,缺点也比较明显。
一方面,结构光容易受环境光干扰,室外成像效果比室内差。
另一方面,随检测距离增加精度会变差,因此,在远距离视野的精度较差。
目前,结构光深度相机在国外比较知名的生产商有:奥比中光,Apple,MicroSoft,Intel。


三种商用方案的粗略对比如\ref{表:深度相机性能对比} 所示。
\begin{table}[h]
	\caption{深度相机性能对比} % 标题
	\centering % 把表居中
	\renewcommand\arraystretch{1.2}
	\setlength{\tabcolsep}{16pt}
	\footnotesize
	\begin{tabular}{cccc} % 四个c代表该表一共四列,内容全部居中
		\toprule % 第一道横线
		相机类型     & 双目深度相机  & TOF深度相机 & 结构光深度相机 \\
		\midrule % 第二道横线 
		成像原理     & 双目特征匹配  & 飞行时间     & 激光散斑编码 \\
		\specialrule{0em}{1pt}{1pt} 
		分辨率       & 高           & 低          & 中 \\
		\specialrule{0em}{1pt}{1pt}
		成像精度     & 中           & 中          & 中 \\
		\specialrule{0em}{1pt}{1pt}
		制造成本     & 低           & 高          & 中 \\
		\bottomrule % 第三道横线
	\end{tabular}
	\label{表:深度相机性能对比}
\end{table}

通过对比以上三种方案可以看出,相比于双目视觉方案,结构光方案在环境适应性、近距离场景数据精度、实时性等方面表现较好;
相比于飞行时间方案,结构光方案的功耗更小,技术更成熟,制造成本更低。
因此,本文选择的传感器是Intel公司的Realsense系列的D435i,该款RGB-D相机采用结构光原理。






\subsection{传统的RGB-D语义分割}
传统的语义分割方法指不使用深度学习的方法。这种方法主要分为三个阶段。
在选择特征阶段,这类方法依赖手工设计的特征,比如颜色、纹理、形状等局部的外观属性,和一些局部特征描述子。
在分类输出阶段,将这些特征送入浅层机器学习分类模型,比如支持向量机(Support Vector Machine,SVM)和随机森林等,从而预测类别。
在结果优化阶段,用图模型,比如马尔可夫随机场(Markov Random Fields,MRF)和条件随机场(Conditional Random Fields,CRF),从而优化预测结果。


有一部分研究聚焦于手工特征的设计。
Shotton等人\cite{Shotton08CVPR}设计了一种新型低级特征语义纹理森林进行语义分割。
Scharwächter等人\cite{Scharwächter15IVS}联合处理了颜色、纹理和深度信息,通过随机决策森林快速推断出街景的粗略布局。


由于基于像素点的语义分割方法没有考虑像素与临近像素之间的关系,因此容易产生不一致的结果。
基于超像素的方法在一定程度上缓解了这个问题,该方法将在同一个局部区域中的像素强制预测成相同类别。
Gupta等人\cite{Gupta15IJCV}通过通用性和特异性特征来编码物体的外观和几何结构,将数据集中的超像素分类做为主要的物体类别,用随机森林和SVM分类,进一步提高了语义分割的准确性。


虽然基于超像素的方法对像素特征的提取更加鲁棒,但是超像素中的像素依然不能保持完全一致。概率图模型在一定程度上增强了空间一致性,从而缓解了这个问题。条件随机场提供一个概率框架,将输出间的关系描述为观测特征的函数。
Ladicky等人\cite{Ladicky10ECCV}针对CRF模型定义了一个全局能量函数,它结合了滑动窗口检测器的结果,以及基于像素的低水平一元和成对关系。
Cadena等人\cite{Cadena14ICRA}提出了一种有效的策略来诱导用于推理的CRF的图结构,增强了空间一致性。


传统的语义分割方法虽然取得了一些成果,但是人工设计的特征很难准确地表述复杂的场景环境,并且浅层机器学习分类模型对非线性函数的拟合能力有限。因此,针对较为复杂的分类问题,传统的语义分割方法效果很难提升。



\subsection{基于深度学习的RGB-D语义分割}
Hinton等人\cite{Hinton06Science}在2006年首次提出深度学习(Deep Learning,DL)之后,深度学习快速发展,语义分割研究也取得了突破性进展。与传统的语义分割方法相比,基于深度学习的语义分割方法能利用深度神经网络超强的非线性拟合能力,获取更多、更高级的语义信息来表达图像中的信息。


\subsubsection{基于卷积神经网络的RGB-D语义分割}
卷积神经网络(Convolutional Neural Network,CNN)的出现,极大地提升了语义分割地性能。
Long等人\cite{Long15CVPR}提出的全卷积神经网络(Fully Convolutional Network,FCN),第一次将深度神经网络引入语义分割领域。FCN推广了原有的CNN结构,利用特定的卷积层替换了常规卷积网络的全连接层,使得用于分类的CNN网络被转化为分割网络。
同时,利用转置卷积层实现上采样,使得网络可以执行密集推理并学习到图像中每个像素的语义标签。
此外,FCN可以处理任意大小的图像,语义分割速度也得到显著提升。
由于第一次实现了对图片进行端到端的训练,所以后续关于语义分割的研究几乎都借鉴了全卷积神经网络结构。

针对RGB-D双模态的语义分割, Couprie等人\cite{Couprie14MLR}提出了一种前融合方案。具体做法是将视频流逐帧分解,将彩色图片和深度图片拼连起来构成一个四通道的输入,放入卷积神经网络的四个输入,该方法实现了视频流的RGB-D实时语义分割,但是并没有区分不同模态的输入。
Gupta等人\cite{Gupta14ECCV}对深度数据进行编码为三个通道,包含水平视差(Horizontal disparity),地上高度(Height above groud)和重力夹角(the Angle the pixel’s local surface normal makes with the inferred gravity
direction)等信息,该编码结构加强了对深度信息的利用,但是不足在于需要相机的位姿等额外数据,并且编码计算代价较高。


















\iffalse 
上述策略都是围绕直接融合采集到的图片数据信息而提出的,即前融合策略, 但这类策略缺乏对不同模态数据的特征提取过程,通常无法兼具最优分割性能。

为了避免这一问题,Hazirbas等人在201 6年提出了一种基于双流网络进行间接特征融合的RGBD语义分割网络FuseNet[ 22],该方法设计了一种图像语义分割的编解码结构,通过在深度通道和图像通道之间添加中间特征映射,提升多模态数据的互补,获得了更好的融合效果。
同年,Li等人提出了一种新型(Long Short-Term Memorized Context Fusion,LSTM-CF)的网络模型[ 23],如图1.9所示,该方法通过水平和垂直多向融合实现全局上下文信息的整合,较好的利用了图像和深度的相关性,提高了语义分割的正确率。 


Long等人将其扩展到了RGB-D数据上:训练两个FCN网络,用来分别处理彩色图像和深度图像;再将两个网络的预测求和,得到最终的分类结果。为了提高融合的效率,Cheng等人设计了门控融合层来自动的学习高层的模态特异的特征[ 68]。
层次融合方法[ 66,71–74]可以在多个不同的层融合两个模态的特征。


FuseNet [71]以自底向上方式将多层的深度特征融入到RGB编码器。
RedNet[ 72]扩展了FuseNet,在自顶向下的路径上融合两个模态的多层特征。
SSMA[ 66]提出了一个自监督的模型适应融合机制(SSMA)来融合模态特征的特征,也是在自顶向下的路径上融合多层的特征。


例如,2017年Lee等人提出的RDFN et网络模型[ 24],就是基于当年里程碑式的RefineNet网络[ 25]而推广提出的。该网络较好的继承了残差学习机制, 通过MMFNet模块和多个RefineNet模块跳跃连接方式,细化图像和深度的融合方式,在试验验证中取得了较好的表现。

同年,Qi等人提出一种基于3D图网络的语义分割新网络模型3DGNN[26],通过将二维图像和深度信息融合,得到对应点的空间三维坐标(x,y,z),有效的减少了图片距离误导的错误发生,但在复杂场景或者二维/三维上下文相似场景中,也容易产生一定的误判。

同年,Schneider等人提出[ 27]了一种基于独立分支的RGBD中层特征融合模型,该方法具有较强的扩展性, 可通过简单的调整适用不同模态数据的处理,在公开数据集上保持了较高的性能指标。

针对深度几何信息在固定卷积神经网络上效果有限的问题,201 8年,Wang等人提出了深度感知卷积和深度感知平均池化[ 28]的算子,在不增加网络计算量的前提下,提高模型对两种模态数据的兼顾性,但实验结果表明这类方法对小目标的效果有限,需要在多尺度处理上进行改进。

同年,Jiang等人提出了一个对称的残差编解码网络[ 29],该网络深度很深却解决了细节遗失和梯度消失的问题,在小目标融合的效果上有显著提升。

近三年,基于深度学习的RGBD算法改进主要是针对语义特征提取网络的优化工作。
2019年Jiao等人提出了深度几何信息引入辅助语义分割的网络框架[ 30],该网络将骨干网络的权值相互分享,改进语义特征,并使用跳跃金字塔模块,提升上下文融合效果。

同年,Hu等人融入了注意力机制,提出了三平行分支架构的网络模型ACNet[ 3 1],充分融合深浅层特征,平衡图片和深度的作用比重,如图1.10 所示。

为了进一步充分利用两者互补性,在2020年北大、商汤科技、港大联合提出了一种引入作用于特征分离和聚合的注意力机制的网络模型[ 32],是互补信息重新校准融合。

2021年Chen等人提出的GLPNet网络[ 33]成功地在较深阶段保持两种信息稳定地传播,其原因在于引入局部上下文融合模块(L-CFM)进行动态对齐,引入全局上下文融合模块(G-CFM)进行联合建模。
\fi



 %\cite{Gupta14lECCV}





\subsubsection{基于transformer的RGB-D语义分割}


\subsubsection{基于mamba的RGB-D语义分割}





\section{主要研究内容与论文组织结构}


\subsection{主要研究内容}

\subsection{论文组织结构}




              
\input{data/chap02}
\chapter{基于通道交换机制的跨模态RGB-D语义分割网络}
%此处主要是与上一章节的逻辑联系。要写一页。
语义分割是计算机视觉领域的一个重要研究任务,该任务旨在对给定图像的每一个像素进行分类。
在地铁排爆场景中,语义分割可以帮助机器人在复杂的场景下理解周边场景,进行定位、导航、排爆等任务。

近年来,卷积神经网络(Convolutional Neural Network,CNN)在高分辨率图像处理方面展示了强大的能力。
其中,全卷积网络(Fully Convolutional Network,FCN)是卷积神经网络在语义分割任务的重量级工作。
语义分割在彩色数据集上的迅速发展,随着RGB-D相机的发展,使用RGB-D数据的语义分割任务发展也如火如荼。

彩色数据和深度数据是同质化的图像信息,但是仍然是分属不同模态的数据,其中蕴含着的信息既有本模态的独特性,又有不同模态的互补性。

针对提取模态信息独特性而言,在传统的卷积操作中,卷积操作是为彩色信息设计的,并且在网络的训练过程中会和深度信息共享卷积参数,这样并不利于提取深度信息。

针对提取模态信息互补性而言,如何融合这种互补性信息,继而增强语义分割在RGB-D数据上的性能,是RGB-D语义分割领域的重中之重。
融合策略分为以下三种:前融合、中间融合、决策融合。
前融合就是指在输入进网络时,将彩色信息和深度信息做一个拼接操作,一起输入语义分割的网络之中。
中间融合就是在进行网络生成特征图之后,对生成的特征图使用融合策略,从而加强语义分割网络对特征的提取能力。
决策融合就是在网络进行最终分割结果运算时,将彩色信息的预测结果和深度信息的预测结果进行加强,得到最终的语义分割结构。
这三种融合策略并不是互相排斥的,他们可以共存在同一个分割网络之中。
由于前融合和决策融合较为简单,因此目前的研究重点主要是在中间层融合。

针对以上两个问题,本章首先提出了一种针对RGB-D双模态信息的卷积模块,该模块可以替换传统的卷积操作,应用在RGB-D语义分割网络中。其次,采用三种融合策略,构建一种基于特征图交换机制的中间融合机制,以加强RGB-D语义分割的性能。
具体而言,本章的主要工作包括几个部分:

(1)针对使用RGB-D双模态信息的语义分割网络,设计了一个跨模态卷积(Cross-modal convolution,CMConv),该部分可以更好地提取RGB-D信息,促进分割性能的提升。

(2)针对使用视觉transformer(vision transformer,VIT)的结构,
提出一种特征交换机制(Feature exchange),该结构可以有效地检测并剔除训练过程中的冗余特征信息,从而增强RGB-D语义分割网络的性能。





\section{模型方法}



\subsection{框架结构}



\subsection{跨模态卷积}
彩色信息和深度信息属于不同模态的信息。彩色图片通过对使用RGB色彩空间对颜色进行三通道赋值来表示捕获到的信息,而深度图片通过对捕获到的深度信息进行赋值产生的单通道灰白图片来表示捕获到的信息。因此,从本质上说,这两种数据是不同的。

从其表示信息的原理上来说,RGB值捕获投影图像空间中的光度外观属性,而深度特征编码局部形状信息及其在上下文中的位置信息。
对于同样形状的物体,我们希望网络可以提取出相同的特征。然而使用普通的卷积运算时,由于其位置的不同,提取出的特征是不同的,这阻碍了形状不变性的学习。
但是,也不能因为追求当前层的形状不变性而简单地将位置信息直接舍弃,因为位置信息在具有更大上下文的后续处理过程中会形成形状信息。
与位置相比,形状是物体更固有的属性,与语义联系更为紧密,因而对分割精度更关键。
因此,广泛用于使用彩色数据的卷积运算在处理深度数据时,并不高效。


基于深度特征可以表征形状信息和位置信息的模态特性,本章引入跨模态卷积来处理深度特征,以学习形状信息和位置信息重要性之间的自适应平衡,使网络有机会在必要时更多地关注形状信息,从而有利于RGB-D语义分割任务。

首先,将深度图片产生的补丁(patch)蕴含的形状信息和位置信息分解为两个独立的部分,得到形状分量(shape component)和位置分量(local component)。
深度补丁的平均值描述了该补丁在更大范围内的位置,从而构成了位置分量,表示距离观测点的距离。
而剩余的部分描述补丁的相对变化,描述了底层的几何形状,从而构成了形状分量,表示物体的语义。

然后,引入两个可学习的权值分别处理形状分量和位置分量得到形状核(shape kernal)和位置核(local kernal),最后对形状核和位置核加权组合,形成一个形状感知的补丁,并进一步与一个正常的卷积核进行卷积。
与原始补丁相比,形状感知补丁能够利用形状核自适应学习形状特征,利用位置核平衡形状和位置对最终预测的贡献。
此外,由于形状核和位置核在推理阶段成为常量,将它们融合到下面的卷积核中可以得到一个与普通卷积层相同的网络。


对于一个输入的补丁$\mathbb{P} \in R^{K_h \times K_w \times C_{in}}$,$K_h$和$K_w$是核的空间维度,$C_{in}$表示输入特征映射中的通道数,传统卷积层得到的输出特征
\begin{equation}
	\label{eq:conv}
	\mathbb{F} = Conv(\mathbb{K}, \mathbb{P}),
\end{equation}
其中,$\mathbb{K} \in R^{K_h \times K_w \times C_{in} \times C_{out}}$表示卷积层中核的可学习权值,$C_{out}$表示输出特征映射中的通道数。
$\mathbb{F} \in R^{C_{out}}$的每个元素计算为:
\begin{equation*}
	\mathbb{F}_{c_{out}} = \sum_{i}^{K_h \times K_w \times C_{in}} (\mathbb{K}_{i,c_{out}} \times \mathbb{P}_i).
	\label{eq:conv-f}
\end{equation*}

可以很容易地看出,$\mathbb{F}$通常会随着$\mathbb{P}$的不同值而变化。
假设有两个一样的物体在不同的位置,分别用补丁$\mathbb{P}_1$和补丁$\mathbb{P}_2$表示。
对应的输出特征:$\mathbb{F}_1$ and $\mathbb{F}_2$从卷积层学习:$\mathbb{F}_1 = Conv(\mathbb{K}, \mathbb{P}_1)$, $\mathbb{F}_2 = Conv(\mathbb{K}, \mathbb{P}_2)$。
由于$\mathbb{P}_1$和$\mathbb{P}_2$与观测点的距离并不相同,因此它们的特征通常不同,这可能导致不同的预测结果。
然而实际上,$\mathbb{P}_1$和$\mathbb{P}_2$属于同一个类别,传统的卷积层不能很好地处理这种情况。

但是,这两个补丁的形状是不变量。形状特征用来表征局部特征下的相对深度差异,而这一点却被现有的方法所忽略。
鉴于此,我们建议通过对RGB-D语义分割的形状进行有效建模来填补这一空白。

基于上述分析,本文提出将输入补丁分解为两个分量:描述补丁位置的位置分量(local component)和表示补丁是什么的形状分量(shape component)。

我们将补丁的平均值称为$\mathbb{P}_B$,其相对值称为$\mathbb{P}_S$。
\begin{equation*}
	\begin{aligned}
		& \mathbb{P}_B = m(\mathbb{P}), \\
		& \mathbb{P}_S = \mathbb{P} - m(\mathbb{P}),
	\end{aligned}
\end{equation*}
其中$m(\mathbb{P})$是$\mathbb{P}$ 上的平均函数(在$K_h \times K_w$维上),
$\mathbb{P}_B \in R^{1 \times 1 \times C_{in}}$,$\mathbb{P}_S \in R^{K_h \times K_w \times C_{in}}$。


注意,在等式\ref{eq:conv}中直接卷积的$\mathbb{P}_S$ with $\mathbb{K}$是次优的,因为来自$\mathbb{P}_B$的值有助于跨块的类别区分。
因此,我们的ShapeConv利用两个可学习的权重 $\mathbb{W}_B \in R^{1}$ and $\mathbb{W}_S \in R^{K_h \times K_w \times K_h \times K_w \times C_{in}}$来分别消耗上述两个分量。
然后以逐元素添加的方式组合输出的特征,这形成具有与原始$\mathbb{P}$相同大小的新的形状感知贴片。

\begin{equation}
	\label{eq:shapeconv-patch}
	\begin{aligned}
		\mathbb{F} &= ShapeConv(\mathbb{K}, \mathbb{W}_B, \mathbb{W}_S, \mathbb{P})\\
		&= Conv(\mathbb{K}, \mathbb{W}_B \diamond \mathbb{P}_B + \mathbb{W}_S \ast \mathbb{P}_S)\\
		&= Conv(\mathbb{K}, \textbf{P}_\textbf{B} + \textbf{P}_\textbf{S})\\
		&= Conv((\mathbb{K}, \textbf{P}_\textbf{BS}),
	\end{aligned}
\end{equation}
其中,$\diamond$ and $\ast$分别表示基本乘积和形状乘积算子,其被定义为

\begin{equation}
	\begin{cases}
		\textbf{P}_\textbf{B} = \mathbb{W}_B \diamond \mathbb{P}_B \\
		\textbf{P}_{\textbf{B}_{1,1,c_{in}}} = \mathbb{W}_B \times \mathbb{P}_{B_{1,1,c_{in}}},
	\end{cases}
	\label{eq:base-product-P}
\end{equation}
\begin{equation}
	\begin{cases}
		\textbf{P}_\textbf{S} = \mathbb{W}_S \ast \mathbb{P}_S \\
		\textbf{P}_{\textbf{S}_{{k_h},{k_w},{c_{in}}}} = \sum_{i}^{K_h \times K_w} (\mathbb{W}_{S_{i,{k_h},{k_w},{c_{in}}}} \times \mathbb{P}_{S_{i,{c_{in}}}}),
	\end{cases}
	\label{eq:shape-product-P}
\end{equation}

其中$c_{in}$, $k_h$, $k_w$分别是$C_{in}$, $K_h$, $K_w$维度中的元素的索引。

我们通过$\textbf{P}_\textbf{B}$ and $\textbf{P}_\textbf{S}$的相加来重建形状感知补丁$\textbf{P}_\textbf{BS}$,$\textbf{P}_\textbf{B}$ and $\textbf{P}_\textbf{S}$,这使得它能够被香草卷积层的内核$\mathbb{K}$平滑卷积。
然而,$\textbf{P}_\textbf{BS}$配备了通过两个额外权重学习的重要形状信息,使得卷积层专注于仅使用深度值失败的情况。

第3.1节中提出的形状转换可以有效地利用补丁。
然而,在CNNs中用ShapeConv代替香草卷积层引入了更多的计算成本,这是由于等式3和4中的两个乘积运算。为了解决这个问题,我们提出将这两个操作从补丁转移到内核
\begin{equation*}
	\begin{aligned}
		\begin{cases}
			\textbf{K}_\textbf{B} = \mathbb{W}_B \diamond \mathbb{K}_B \\
			\textbf{K}_{\textbf{B}_{1,1,c_{in},c_{out}}} = \mathbb{W}_B \times \mathbb{K}_{B_{1,1,c_{in},c_{out}}},
		\end{cases}
		\label{eq:base-product-K}
	\end{aligned}
\end{equation*}
\begin{equation*}
	\begin{aligned}
		\begin{cases}
			\textbf{K}_\textbf{S} = \mathbb{W}_S \ast \mathbb{K}_S \\
			\textbf{K}_{\textbf{S}_{{k_h},{k_w},{c_{in}},{c_{out}}}} = \sum_{i}^{K_h \times K_w} (\mathbb{W}_{S_{i,{k_h},{k_w},{c_{in}}}} \times \mathbb{K}_{S_{i,{c_{in}},{c_{out}}}}),
		\end{cases}
		\label{eq:shape-product-K}
	\end{aligned}
\end{equation*}

其中$\mathbb{K}_B \in R^{1 \times 1 \times C_{in} \times C_{out}}$ and $\mathbb{K}_S \in R^{K_h \times K_w \times C_{in} \times C_{out}}$分别表示核的基分量和形状分量,$\mathbb{K} = \mathbb{K}_B + \mathbb{K}_S$。因此,我们将方程2的形状转换重新形式化为以下:
\begin{equation}
	\label{eq:shapeconv-kernel}
	\begin{aligned}
		\mathbb{F} &= ShapeConv(\mathbb{K}, \mathbb{W}_B, \mathbb{W}_S, \mathbb{P})\\
		&= Conv(\mathbb{W}_B \diamond m(\mathbb{K})+ \mathbb{W}_S \ast (\mathbb{K} - m(\mathbb{K})), \mathbb{P})\\
		&= Conv(\mathbb{W}_B \diamond \mathbb{K}_B+ \mathbb{W}_S \ast \mathbb{K}_S, \mathbb{P})\\
		&= Conv(\textbf{K}_\textbf{B} + \textbf{K}_\textbf{S}, \mathbb{P})\\
		&= Conv(\textbf{K}_\textbf{BS}, \mathbb{P}),
	\end{aligned}
\end{equation}

其中 $m(\mathbb{K})$是$\mathbb{K}$上的平均函数(在$K_h \times K_w$维度上)。
我们要求$\textbf{K}_\textbf{BS} = \textbf{K}_\textbf{B} + \textbf{K}_\textbf{S}$, $\textbf{K}_\textbf{BS} \in R^{K_h \times K_w \times C_{in} \times C_{out}}$。

事实上,ShpeConv的两个公式,即,等式\ref{eq:shapeconv-patch}和等式\ref{eq:shapeconv-kernel}在数学上是等价的,即,
\begin{equation}
	\begin{aligned}
		\mathbb{F} &= ShapeConv(\mathbb{K}, \mathbb{W}_B, \mathbb{W}_S, \mathbb{P})\\
		&= Conv(\mathbb{K}, \textbf{P}_\textbf{BS})\\
		&= Conv(\textbf{K}_\textbf{BS}, \mathbb{P}),
	\end{aligned}
\end{equation}

\begin{equation}
	\begin{aligned}
		\mathbb{F}_{c_{out}} &= \sum_{i}^{K_h \times K_w \times C_{in}} (\mathbb{K}_{i,{c_{out}}} \times \textbf{P}_{\textbf{BS}_{i}})\\
		&= \sum_{i}^{K_h \times K_w \times C_{in}} (\textbf{K}_{\textbf{BS}_{i,{c_{out}}}} \times \mathbb{P}_{i}),
	\end{aligned}
\end{equation}


推论阶段。在推理过程中,由于$\mathbb{W}_B$ and $\mathbb{W}_S$这两个附加权重变为常数,因此我们可以将它们融合成$\textbf{K}_\textbf{BS}$,如\ref{fig:ShapeConv}(c)所示$\textbf{K}_\textbf{BS} = \mathbb{W}_B \diamond \mathbb{K}_B+ \mathbb{W}_S \ast \mathbb{K}_S$。
$\textbf{K}_\textbf{BS}$与等式\ref{eq:conv}中的$\mathbb{K}$共享相同的张量大小,因此,我们的ShapeConv实际上与\ref{fig:ShapeConv}(a)中的香草卷积层相同。换句话说,当用ShapeConv代替香草卷积时,不会引入额外的推理时间。


%\subsection{特征交换机制}
如引言中所讨论的,深度多模态融合方法主要可以分为基于聚合的融合和基于融合的融合[4]。
由于模态内处理的弱点,最近的基于聚合的工作执行特征融合,同时仍然保持所有模态的子网络[12,30]。
此外,[19]指出,熔合的性能受到选择熔合哪一层的高度影响。
基于对齐的融合方法通过应用相似性规则来对齐多模态特征,其中最大均值差异MMD)[16]通常用于测量。
然而,仅仅关注统一整个分布可能会忽略每个领域/模态中的特定模式[6,44]。
因此,[47]提供了一种可以缓解这一问题的方法,该方法将模态共同特征相关联,同时保持模态特定信息。
还有一部分基于调制的多模态学习文献[11,13,46]。
不同于这些类型的融合方法,我们提出了一种新的融合方法,通过通道交换,这可能享有充分的模型间的相互作用和模态内学习的保证。
使用BN缩放因子来评估CNN通道重要性的想法已经在网络修剪[33,49]和表示学习[40]中进行了研究。
[33]对缩放因子实施了101范数惩罚,并显式地修剪掉满足稀疏性标准的过滤器。
在这里,我们将这个想法作为一种自适应工具来确定在哪里交换和融合。
CBN [46]通过以另一种模态为条件调制一种模态的BN来执行跨模态消息传递,这显然不同于我们在不同模态之间直接交换信道以进行融合的方法。
ShuffleNet [53]提出在多个组之间移动一部分信道,以在轻量级网络中进行有效传播,这类似于我们交换信道进行消息融合的想法。
然而,虽然我们的论文的动机是非常不同的,但交换过程是由BN缩放因子自决定的,而不是ShuffleNet中的随机交换。








Transformer最初在自然语言社区中被广泛研究为非递归序列模型[40],并且很快被扩展以使视觉语言任务受益。最近,许多研究进一步采用变压器进行计算机视觉任务,具有良好的适应性架构和优化时间表。因此,视觉跨以前的变体已经在许多单模视觉任务中显示出巨大的潜力,例如分类[6,21]、分割[44,47]、检测[3,8,22,48]、图像生成[16]。然而,直到这项工作的日期,尝试扩展视觉转换器,以处理多模态数据仍然很少。当引入具有复杂对齐关系的多模态数据时,对模型体系结构的融合方案设计提出了很大的挑战。要回答的关键问题是,不同模态的特征之间的交互应如何以及在何处发生。已有几种基于变换器的视觉语言融合方法,VL-BERT [37]和ViLT [17]中所述的方法。在这些方法中,视觉和语言标记在每个Transformer层之前直接级联,使得整体架构与原始Transformer非常相似。这种融合通常是比对不可知的,这表明没有明确地利用模态间比对。我们还尝试将类似的融合方法应用于多模态视觉任务(第4)、第四章。不幸的是,这种直观的Transformer融合不能带来有希望的增益,或者甚至可能导致比单模态对应物更差的性能,这主要是由于没有充分利用模态间的相互作用。也有几种尝试用于融合多种视觉模态。例如,TransFuser [26]利用Transformer模块来连接图像的CNN主干和LiDAR点。与已有的试验不同,本文旨在寻求一种有效且通用的方法,将多个单模态变压器组合起来,并在模型中插入模态间的对齐。这项工作有利于多模态数据的学习过程,同时利用模态间对齐。这种对准在许多视觉任务中自然可用,例如,利用摄像机内函数/外函数,世界空间点可以被投影并且对应于摄像机平面上的像素。与不可知论融合(Sec.3.1),该martaware融合明确涉及的对齐关系,
不同的模式。然而,由于在Transformer中引入了模态间投影,因此对准感知融合可能会极大地改变原始模型结构和数据流,这可能会破坏单模态架构设计的成功或预训练期间习得的注意力。因此,可能必须为多模投影和融合确定“正确的”层/记号/通道,并且还必须为新模型重新设计架构或重新调整优化设置。为了避免处理这些具有挑战性的问题并继承原始单模设计的大部分,我们提出了多模式令牌融合,称为TokenFusion,它自适应地并有效地融合多个单模变换器。我们的TokenFusion的基本思想是修剪多个单模态变换器,然后重新利用修剪后的单元进行多模态融合。我们对每个单模态Transformer应用单独的修剪,并且每个修剪的单元由来自其他模态的投影对准特征代替。假设该融合方案对原始的单模态变压器具有有限的影响,因为它保持了重要单元的相对注意关系。TokenFusion在允许多模态转换器继承来自单模态预训练的参数方面也被证明是上级的,在ImageNet上。展示优势


融合视觉变形。与多模态数据的深度融合一直是一个重要的主题,它可能通过利用多个输入源来提高性能,并且还可能进一步释放变压器的力量。然而,将多个现成的单变压器联合收割机组合在一起,同时保证这种组合不会影响其精心设计的单模态设计,这是具有挑战性的。信号装置.[2]以及[20]利用变换器处理连续的视频帧,用于空间-时间对准,并通过使多个帧相关来捕获细粒度模式。关于多模态数据,[26,41]利用Transformer模块的动态特性来联合收割机CNN主干,以融合红外/可见光图像或LiDAR点。[9]将从粗到精的经验从CNN融合方法扩展到用于图像处理任务的变换器。[14]采用变换器联合收割机高光谱图像进行简单的特征拼接。[24]在图像补片和音频频谱图补片之间插入中间标记作为瓶颈以隐式地学习模态间对准。然而,这些工作与我们的工作不同,因为我们希望构建一个通用的融合管道,用于组合现成的视觉转换器,而无需重新设计其结构或重新调整其优化设置,同时明确利用模态间的对齐关系。


Vision Transformers的基本融合

假设我们有第$i$个输入数据$\bm{x}^{(i)}$,它包含$M$个模态:$\bm{x}^{(i)}=\{\bm{x}_m^{(i)}\in\mathbb{R}^{N\times C}\}_{m=1}^M$,其中$N$ and $C$分别表示令牌和输入通道的数量。
为了简单起见,我们将在接下来的部分中省略下标$^{(i)}$。
深度多模态融合的目标是确定一个多层模型 $f(\bm{x})$,期望其输出尽可能接近目标$\bm{y}$。
具体来说,在这项工作中,$f(\bm{x})$是近似基于transformerbased网络架构。
假设模型总共包含$L$ 层,我们表示第$l$层($l=1,\ldots,L$)为$\bm{e}^l=\{\bm{e}^l_m\in\mathbb{R}^{N\times C'}\}_{m=1}^M$,其中$C'$表示范围内的层的特征通道的数量。最初,使用$\bm{e}_m^1$的线性投影来获得$\bm{x}_m$,这是一种广泛采用的对输入标记(例如,图像块)进行矢量化的方法,使得第一Transformer层可以接受标记作为输入。





我们对输入模态使用不同的变换器,并将$f_m(\bm{x})=\bm{e}_m^{L+1}$表示为第$m$个Transformer的最终预测。给定第$m$个模态的令牌特征$\bm{e}_m^{l}$,第$l$层计算





翻译到这里 懂?
1这是第三章节 懂?1
还有什么问题?我试试不动了喊你1
\subsection{Vision Transformer 的基本融合}
\label{subsec:intuitive_fusion}
假设我们有第$i$个输入数据$\bm{x}^{(i)}$它包含$M$个模态: $\bm{x}^{(i)}=\{\bm{x}_m^{(i)}\in\mathbb{R}^{N\times C}\}_{m=1}^M$,其中,$N$ 和$C$分别表示令牌和输入通道的数量。 
为了简单起见,我们将在接下来的部分中省略下标$^{(i)}$ 。
深度多模态融合的目标是确定一个多层模型 $f(\bm{x})$,期望其输出尽可能接近目标 $\bm{y}$ 。 
具体来说,在这项工作中,$f(\bm{x})$是近似基于transformerbased的网络架构。假设模型总共包含$L$层, 我们表示第$l$层($l=1,\ldots,L$)为$\bm{e}^l=\{\bm{e}^l_m\in\mathbb{R}^{N\times C'}\}_{m=1}^M$, 其中$C'$表示范围内的层的特征通道的数量。
最初,使用$\bm{x}_m$的线性投影来获得$\bm{e}_m^1$,这是一种广泛采用的对输入标记(\emph{例如},图像块)进行矢量化的方法,使得第一Transformer层可以接受标记作为输入。

我们对输入模态使用不同的变换器,并将$f_m(\bm{x})=\bm{e}_m^{L+1}$表示为第$m$个Transformer的最终预测。
给定第$m$个模态的令牌特征$\bm{e}_m^{l}$,第$l$层计算
\begin{equation}
	\label{eq:token-feature}
	\hat{\bm{e}}_m^l=\text{MSA}\big(\text{LN}(\bm{e}_m^{l})\big),\; \bm{e}_m^{l+1}=\text{MLP}\big(\text{LN}(\hat{\bm{e}}_m^l)\big),
\end{equation}
其中$\text{MSA}$, $\text{MLP}$,和 $\text{LN}$表示多头自注意、多层感知和层归一化,
$\hat{\bm{e}}^l_m$ 代表MSA的输出。 

在视觉任务的多模态融合过程中,不同模态的对齐关系可以显式地可用。例如,像素位置通常用于确定图像深度相关性;并且相机内函数/外函数在将3D点投影到图像中时很重要。基于对齐信息的参与,我们考虑了以下两种Transformer融合方法。

对齐不可知融合不明确使用模态之间的对齐关系。该算法期望从大量的数据中隐式地学习到对齐。对齐不可知融合的一种常用方法是直接拼接多模态输入标记,广泛应用于视觉语言模型。类似地,用于第$l$层的输入特征$\bm{e}_l$也是不同模态的令牌式级联。尽管对准不可知的融合是简单的并且可以对原始Transformer模型具有最小的修改,但是很难直接受益于已知的多模态对准关系。

\label{sec:对齐感知融合}明确地利用模态间对齐。例如,这可以通过选择对应于相同像素或3D坐标的标记来实现。假设$\bm{x}_m[n]$是第$m$个模态输入$\bm{x}_m$的第$n$个令牌,其中$n=1,\cdots,N_m$。我们将从第$m$个模态到第$m'$个模态的“标记投影”定义为:
\begin{equation}
	\label{eq:token-projection}
	\mathrm{Proj}^\text{T}_{m'}(\bm{x}_{m}[n_{m}])=h(\bm{x}_{m'}[n_{m'}]),
\end{equation}
其中$h$可以简单地是身份函数(对于同质模态)或浅多层感知(对于异质模态)。当考虑整个$N$个token时,我们可以方便地将“模态投影”定义为token投影的串联:
projections:\vskip-0.2in
\begin{equation}
	\label{eq:modality-projection}
	{\mathrm{Proj}}^\text{M}_{m'}(\bm{x}_{m})=\big[\mathrm{Proj}^\text{T}_{m'}(\bm{x}_{m}[1]); \cdots; \mathrm{Proj}^\text{T}_{m'}(\bm{x}_{m}[N])\big].
\end{equation}

\ref{eq:modality-projection} 仅示出了输入侧的融合策略。我们还可以通过投影和聚合特征嵌入$\bm{e}_m$,在不同模态特定模型之间执行中间层或多层融合,这可能实现更多样化和更精确的特征交互。
然而,随着基于变换器的模型的复杂性的增长,搜索仅用于两种模态 (\emph{例如} 2D和3D检测变换器)的最佳融合策略(\emph{例如} 应用投影和聚集的层和标记)可能会增长为极难解决的问题。 
为了解决这一问题, 我们在第2.3节中提出了多模式令牌融合 \ref{subsec:mix_transformers}.



\subsection{多模式令牌融合}
\label{subsec:mix_transformers}



正如 \ref{sec:intro}描述的, 多模态令牌融合 (TokenFusion) 首先修剪单模态变换器,并进一步重新利用修剪后的单元进行融合。通过这种方式,原始的单模态变压器的信息单元被假定为在很大程度上被保留,而多模态的相互作用可以被涉及以提高性能。

如之前在\cite{DBLP:journals/corr/abs-2106-02034}中所示,视觉变换器的标记可以在保持性能的同时以分层方式被修剪。类似地,我们可以通过采用评分函数$s^l(\bm{e}^{l})=\text{MLP}(\bm{e}^{l})\in[0,1]^N$来选择较少信息的标记, 该评分函数动态地预测第$l$层和第$m$模态的标记的重要性。
为了在$s^l(\bm{e}^{l})$上实现反向传播,我们重新公式化方程中的MSA输出$\hat{\bm{e}}_m^l$ in \ref{eq:token-feature} as 
\begin{equation}
	\label{eq:msa-output-score}
	\hat{\bm{e}}_m^l=\text{MSA}\big(\text{LN}(\bm{e}_m^{l})\cdot s^l(\bm{e}_m^{l})\big).
\end{equation}

我们使用$\mathcal{L}_m$来表示第$m$个模态的任务特定损失。为了修剪无信息的标记,我们进一步在$s^l(\bm{e}_m^{l})$上添加标记式修剪损失($l_1$-norm)。因此,用于优化的总损失函数被导出为:
\vskip-0.1in
\begin{equation}
	\label{eq:overall-loss}
	\mathcal{L}=\sum_{m=1}^M\Big(\mathcal{L}_m+\lambda\sum_{l=1}^L\big|s^l(\bm{e}_m^{l})\big|\Big),
\end{equation}
其中$\lambda$ 是用于平衡不同损耗的超参数。

对于特征$\bm{e}_m^l\in\mathbb{R}^{N\times C'}$,令牌式剪枝从所有$N$个令牌中动态检测不重要的令牌。改变不重要的标记或用其他嵌入替换它们,预计对其他信息标记的影响有限。
因此,我们提出了一种用于多模态变换器的标记融合过程,因此,我们为多模态变换器提出了一种标记融合程序,用其他模态的标记投影(定义见第 \ref{sec:alignment-aware-fusion}节)替代不重要的标记。
由于修剪过程是动态的,{\emph{i.e.}},即,在输入特征的条件下,融合过程也是动态的。该过程在每个Transformer层之前执行令牌替换,因此第$l$层的输入特征,\emph{i.e.},$\bm{e}_m^l$,被重新表述为\vskip-0.2in
\begin{align}
	\label{eq:mix-process}
	\bm{e}_m^l&=\bm{e}_m^l\odot\mathbb{I}_{s^l(\bm{e}_m^{l})\ge\theta}
	+{\mathrm{Proj}}^\text{M}_{m'}(\bm{e}_m^l)\odot\mathbb{I}_{s^l(\bm{e}_m^{l})<\theta},
\end{align}
其中$\mathbb{I}$是一个断言下标条件的指示符,因此它输出一个掩码张量$\in\{0,1\}^N$;参数$\theta$是一个小阈值(我们在实验中采用$10^{-2}$);运算符$\odot$ 表示逐元素乘法。

In \ref{eq:mix-process}, 如果只有两个模态作为输入,则$m'$将仅仅是除$m$之外的另一模态。对于两个以上的模态,我们将标记预先分配为$M-1$个部分,每个部分都与其他模态中的一个绑定,而不是与它绑定。此预分配的更多细节将在\ref{subsec:homogeneous-modalities}. 


\subsection{剩余位置对准} 
\label{subsec:rpa}
直接替换代币将冒着完全破坏其原始位置信息的风险。
因此,模型仍然可以忽略来自另一模态的投影特征的对齐。
为了缓解这个问题,我们采用了残余位置对齐(RPA),利用位置嵌入(PE)的多模式对齐。
图\ref{pic:framework} and 图\ref{pic:framework-hete} 所示。
稍后将详细介绍,RPA 的关键理念在于向后续层注入等效 PE。
此外,PE的反向传播在第一层之后停止,这意味着在整个训练过程中,仅保留第一层处的PE的梯度,而对于其余的层则冻结。以这种方式,PE服务于对齐多模态令牌的目的,而不管原始令牌的替换状态。总之,即使替换了标记,我们仍然保留从另一个模态添加到投影特征的原始PE。 



\begin{figure}[t!]
	\centering
	\hskip0.07in
	\includegraphics[scale=0.42]{figures/mixt-ho.pdf}
	\caption[]{以RGB和深度为例,介绍了一种用于同质模态的TokenFusion框架。这两种模态都被发送到共享的Transformer,其中还具有共享的位置嵌入。}
	\label{pic:framework}
	\vskip-0.1in
\end{figure}

\begin{figure*}[t!]
	\centering
	\includegraphics[scale=0.42]{figures/mixt-he.pdf}
	\caption[]{用于点云和图像的异构模态的TokenFusion框架。两种模态都被发送到具有单独位置嵌入的单独Transformer模块。需要额外的模态间投影(Proj),这与同质模态的融合不同
	}
	\label{pic:framework-hete}
	\vskip-0.1in
\end{figure*}





%\begin{figure}[htp]
%	\centering
%	\includegraphics{picmain}
%	\caption{图 1.1 名称}
%\end{figure}

%\begin{table}[htp]
%	\centering
%	\begin{minipage}[t]{0.8\linewidth} % 如果想在表格中使用脚注,minipage是个不错的办法
%		\caption[表 1.1 名称]{}
%		\begin{tabular*}{\textwidth}{lp{10cm}}
%			\toprule[1.5pt]
%			{\hei 列1} & {\hei 列2} \\
%			\midrule[1pt]
%			&  \\
%			& \\
%			& \\
%			& \\
%			& \\
%			& \\
%			\bottomrule[1.5pt]
%		\end{tabular*}
%	\end{minipage}
%\end{table}

%\subsubsection{(1.1.1.1 题目)}


\chapter{非对称结构的实时跨模态RGB-D语义分割网络}



此处主要是与上一章节的逻辑联系。要写一页。

本章基于非对称的Transformer+Transformer结构,提出了一种轻量化的实时跨模态RGB-D语义分割网络CMFormer。具体而言,本章的主要工作包括几个部分:

(1)提出了一种基于非对称结构的跨模态RGB-D语义分割框架,分别处理RGB信息和深度信息的语义分割。
在RGB分支的VIT(Vision Transformer)中使用top-k稀疏注意力(top-k Sparse Attention, top-k SA),用以减少注意力机制计算时的信息冗余,降低模型大小,提高模型计算速度。
在RGB分支的VIT(Vision Transformer)使用轻量级的mix-transformer处理深度特征,该结构在处理深度信息的同时极大的压缩了模型大小。
	
(2)跨模态融合模块中使用特征选择模块,用以提取RGB模态和深度模态的有效信息。使用基于跨模态注意力引导的特征融合模块,用以融合RGB模态和深度模态,最后将融合的模态替换深度模态的原有信息。

(3)使用轻量级MLP解码器来解码浅层特征的语义信息,实现语义分割。



\section{模型方法}



\subsection{框架结构}
本章设计的CMFormer算法采用双分支结构处理RGB信息和深度信息,通过四个阶段的降采样对不同尺度的信息进行特征编码,采取中间层融合策略对不同尺寸的不同模态的信息进行融合,最后对融合的特征图进行解码实现RGB-D语义分割。

本章的算法主要有以下四个部分:基于top-k transformer的RGB信息处理分支、基于mix-transformer的深度信息处理分支、基于注意力引导的跨模态信息融合结构和轻量级的MLP解码器。如所示。
网络以三通道的RGB信息和三通道的深度信息作为输入,top-k transformer负责处理RGB信息,mix-transformer负责处理深度信息。由于第一阶段浅层特征比较明显,因此不进行特征融合,但是在之后的阶段都进行特征融合,基于注意力引导的跨模态信息融合结构负责融合不同阶段不同尺度的彩色信息和深度信息,将得到的融合信息放入深度信息处理通道。轻量级的MLP解码器负责第四阶段后的解码。


总体结构如\ref{图:efficient} 所示。
\begin{figure}[h]
	\centering
	\includegraphics[width=\textwidth]{figures/efficient.png}
	\caption{efficient}
	\label{图:efficient}
\end{figure}


\subsection{基于top-k区域注意力机制的彩色信息处理分支}
常见的视觉transformer(vision transformer, VIT)在进行图像处理时,会先把图片切割成小块,然后将这些小块展平为序列作为输入。
VIT使用的是自注意力机制(Multi-Head Self-Attention, MHSA)。MHSA在进行计算的时候,需要计算输入序列中每个序列与其他所有序列之间的相似度,此时产生的计算复杂度为$O(N^2)$,其中, $N$是序列的长度,由小块的尺寸决定。
因此,在图片分辨率较大的时候,序列长度也较长,但是序列长度二次方增长的复杂度会导致计算量的急剧增长。
此外,在语义分割任务中,序列的分类更多的跟其周围的序列相关,不是所有的序列都有必要和其他的序列进行注意力的计算。

针对上述问题,本章算法提出基于top-k区域注意力机制(top-k regions attention,TRA)设计了处理彩色信息的视觉transformer。
top-k区域注意力原理如下:首先,将图片切割成包含若干个小块的区域。然后,计算区域之间的相似度,保留相似度最高的k个区域。最后,对区域内的小块使用稀疏注意力机制。top-k区域注意力机制的引入有效地减少了序列长度,降低了计算量。


划分区域。给定一个二维的特征图 $\mathbf{X} \in \mathbb{R}^{H \times W \times C}$,
将其划分为 $S \times S$个非不重合的区域,使每个区域包含 $\frac{HW}{S^2}$ 特征向量,
这时,$\mathbf{X}$重塑为$\mathbf{X}^r \in \mathbb{R}^{S^2 \times \frac{HW}{S^2} \times C}$,
继而通过线性投影可以得到$\mathbf{Q}, \mathbf{K}, \mathbf{V} \in \mathbb{R}^{S^2 \times \frac{HW}{S^2} \times C}$:
\begin{equation}
	\mathbf{Q} = \mathbf{X}^r \mathbf{W}^q, \; \;
	\mathbf{K} = \mathbf{X}^r \mathbf{W}^k, \; \;
	\mathbf{V} = \mathbf{X}^r \mathbf{W}^v,
\end{equation}
其中, $\mathbf{W}^q, \mathbf{W}^k, \mathbf{W}^v \in \mathbb{R}^{C \times C}$ 分别是$\mathbf{Q}, \mathbf{K}, \mathbf{V}$的投影权重。

选择top-k区域。通过构造有向图来选择与给定区域相关度排列最高的k个区域。
对之前得到的$\mathbf{Q}, \mathbf{K}$分别求取区域内的均值得到区域层面的$\mathbf{Q}^r, \mathbf{K}^r \in \mathbb{R}^{S^2 \times C}$,将$\mathbf{Q}^r$和$\mathbf{K}^r$的转置进行矩阵乘法运算可以得到邻接矩阵$\mathbf{A}^r \in \mathbb{R}^{S^2 \times S^2}$。
\begin{equation}
	\mathbf{A}^r = \mathbf{Q}^r (\mathbf{K}^r)^T. 
\end{equation}

邻接矩阵$\mathbf{A}^r$是两个区域语义相关的度量。
然后,通过为每个区域保留关联度最高的k个区域,得到稀疏邻接矩阵$\mathbf{I}_r \in \mathbb{N}^{S^2 \times k}$。
\begin{equation}
	\mathbf{I}^r = \mathrm{topkIndex}(\mathbf{A}^r).
\end{equation}
其中,$\mathbf{I}^r$ 的第$i$行包含了$k$个最相关区域的索引。

计算稀疏注意力。得到了稀疏邻接矩阵$\mathbf{I}_r$,就可以使用稀疏注意力。
首先,合并k个相关区域的$\mathbf{K}, \mathbf{V}$。
\begin{equation}
	\mathbf{K}^{g} = \mathrm{gather}(\mathbf{K}, \mathbf{I}^r), \; \; \mathbf{V}^g = \mathrm{gather}(\mathbf{V}, \mathbf{I}^r),
\end{equation}
其中,$\mathbf{K}^g, \mathbf{V}^g \in \mathbb{R}^{S^2 \times \frac{kHW}{S^2} \times C}$由k个相关区域对应参数合并产生。

然后,对收集的键值对使用稀疏注意力。
\begin{equation}
	\mathbf{O} = \mathrm{Attention}(\mathbf{Q}, \mathbf{K}^g, \mathbf{V}^g).
\end{equation}


复杂度的理论计算。
相比与普通注意力机制的复杂度$O((HW)^2)$,TRA的复杂度降低到了$O((HW)^\frac{4}{3})$。
TRA的复杂度计算包括三个部分:划分区域、选择top-k区域、计算稀疏注意力。
因此,总体的复杂度
\begin{equation}\label{eq:complexity}
	\begin{aligned}
		\mathrm{FLOPs} &= \mathrm{FLOPs}_{region} + \mathrm{FLOPs}_{top-k} + \mathrm{FLOPs}_{attention} \\
		&= 3HWC^2 + 2 (S^2)^2 C + 2HW k \frac{HW}{S^2} C \\
		&= 3HWC^2 + C (2S^4 + \frac{k(HW)^2}{S^2} + \frac{k(HW)^2}{S^2}) \\
		&\geq 3HWC^2 + 3C(2S^4 \cdot \frac{k(HW)^2}{S^2} \cdot \frac{k(HW)^2}{S^2})^\frac{1}{3} \\
		&= 3HWC^2 + 3Ck^\frac{2}{3}(2HW)^\frac{4}{3}
	\end{aligned}
\end{equation}

其中,$C$是特征映射的通道数,$k$是参与的区域数。公式中的放缩使用了均值不等式,当且仅当$2S^4 = \frac{k(HW)^2}{S^2}$时等式成立。
因此:
\begin{equation}\label{eq:condition}
	S = (\frac{k}{2}(HW)^2)^\frac{1}{6}.
\end{equation}

当根据公式\ref{eq:condition}划分区域大小时,TRA的复杂度可以降低到$O((HW)^\frac{4}{3})$。

TRA的伪代码如XX所示。
\begin{algorithm}[h]
	\caption{TRA}
	\label{alg:code}
	\begin{algorithmic}[1]
		\REQUIRE 特征图尺寸是(H, W, C)。
		$k$ 是区域数量。
		$S$ 区域数量的算术平方根。
		\ENSURE TRA处理后的特征图(H, W, C)。		
		\STATE $x = \mathit{patchify}(input, \mathit{patch\_size}=H//S)$
		\STATE $query, key, value = \mathit{linear\_qkv}(x).chunk(3, \mathit{dim}=-1)$
		\STATE $query\_r, key\_r = query.mean(\mathit{dim}=1), key.mean(\mathit{dim}=1)$
		\STATE $A_r = \mathit{mm}(query\_r, key\_r.transpose(-1, -2))$
		\STATE $I_r = \mathit{topk}(A_r, k).index$
		\STATE $key\_g = \mathit{gather}(key, I_r)$  
		\STATE $value\_g = \mathit{gather}(value, I_r)$  
		\STATE $A = \mathit{bmm}(query, key\_g.transpose(-2, -1))$
		\STATE $A = \mathit{softmax}(A, \mathit{dim}=-1)$
		\STATE $output = \mathit{bmm}(A, value\_g) + \mathit{dwconv}(value)$
		\STATE $output = \mathit{unpatchify}(output, \mathit{patch\_size}=H//S)$
	\end{algorithmic}
\end{algorithm}

TRA过程如图如\ref{图:top-k区域注意力机制} 所示。
\begin{figure}[h]
	\centering
	\includegraphics[width=\textwidth]{figures/top-k区域注意力机制.pdf}
	\caption{top-k区域注意力机制}
	\label{图:top-k区域注意力机制}
\end{figure}

基于TRA的视觉transofrmer如图如\ref{图:基于top-k区域注意力机制的RGB处理分支}所示。
\begin{figure}[h]
	\centering
	\includegraphics[width=\textwidth]{figures/基于top-k区域注意力机制的RGB处理分支.pdf}
	\caption{基于top-k区域注意力机制的RGB处理分支}
	\label{图:基于top-k区域注意力机制的RGB处理分支}
\end{figure}




\subsection{基于稀疏自注意力的深度信息处理分支}
VIT依靠多头自注意力机制强大的特征提取能力,对输入信息进行高效特征提取,在语义分割领域取得了很好的进展。
但是,不同于三通道的彩色信息,深度信息是单通道的,所以其包含的有效信息比彩色信息更少。

针对这个问题,本文在深度信息处理分支,改进VIT使用的自注意力机制,
使用稀疏自注意力构建更高效的transformer,获取更加轻量化的语义分割模型。


%\textbf{(1)稀疏自注意力。}
%
%在本章,对自注意力机制进行改进,使用稀疏自注意力机制,组成轻量级的transformer块嵌入到编码器中,用较低的计算成本来完成语义分割任务。

\textbf{(1)重叠补丁嵌入。}
对于一个输入的图片来说,ViT使用的不重叠的补丁嵌入操作,将一个$N \times N \times 3$的补丁统一为一个$1 \times 1 \times C$的向量。
这种操作可以很容易地将一个$2 \times 2 \times C_i$的特征统一为$1 \times 1 \times C_{i+1}$的向量,从而获得分层特征映射。
因此,我们可以将不同的层次特征不断缩小,从而获得预期大小尺寸的特征映射。
但是,不重叠的补丁嵌入不能保证补丁的局部连续性。
因此,我们使用重叠补丁嵌入。
通过定义补丁大小、两个相邻补丁之间的步幅和填充大小,可以产生与不重叠补丁嵌入一样大小的特征。
重叠补丁嵌入如\ref{图:补丁嵌入} 所示。
\begin{figure}[h]
	\centering%
	\subfloat[非重叠补丁嵌入]{%
		\label{fig:rescue_1}
		\includegraphics[width=0.5\textwidth]{figures/非重叠补丁嵌入.png}}
	\subfloat[重叠补丁嵌入]{%
		\label{fig:rescue_3}
		\includegraphics[width=0.5\textwidth]{figures/重叠补丁嵌入.png}}
	\vspace{-1em}
	\caption{补丁嵌入}
	\label{图:补丁嵌入}
\end{figure}



\textbf{(2)稀疏自注意力。}
传统的自注意力机制如下所示:
\begin{equation}
	{\rm Attention}({Q}, {K}, {V}) = {\rm Softmax}(\frac{{Q}{K}^\mathsf{T}}{\sqrt{d_{head}}}){V}.
	\label{eqn:mhsa}
\end{equation}

本文采用XX介绍的序列简约算法,该算法使用稀疏因子R来缩减序列的长度。如下所示:
\begin{equation}
	\begin{aligned}
		&{\hat{K}} = {\text{Reshape}(\frac{N}{R},C \cdot R)(K)}\\
		&{K} = {\text{Linear}(C \cdot R, C)({\hat{K}}),}
	\end{aligned}
	\label{eqn:reduce}
\end{equation}

其中,$K$是待稀疏的序列,$R$是一个固定的稀疏因子。在实验中的阶段1到阶段4,$R$被设定为[64,~16,~4,~1]。
在上述公式中,第一个公式将$K$的形状由$N \times C$重塑为$\frac{N}{R}\times (C \cdot R)$,
第二个公式将重塑的$K$线性操作,将其形状展开为$\frac{N}{R} \times C$。
因此,该操作可以使自注意机制的复杂性从$O(N^2)$降低到$O(\frac{N^2}{R})$。
如\ref{图:稀疏自注意力} 所示。
\begin{figure}[h]
	\centering
	\includegraphics[width=\textwidth]{figures/稀疏自注意力.png}
	\caption{稀疏自注意力}
	\label{图:稀疏自注意力}
\end{figure}






\textbf{(3)高效的前馈网络。}
在VIT中使用的传统Transformer模型中,
由于注意力机制并没有考虑图片小块的先后顺序信息,
因此需要通过位置编码(Positional Encoding,PE)这种方式把前后的位置信息加在输入的图片小块上,
这样让Transformer保留图片小块的位置信息,可以提高模型对序列的理解能力。
然而,位置编码的分辨率是固定的。
因此,当测试分辨率与训练分辨率不同时,需要对位置编码进行插值,这往往会导致预测准确性下降。


为了缓解这个问题,
本文引入Mix-FFN(Mixed Feed-Forward Network),
该结构考虑了零填充对泄漏位置信息的影响,通过在前馈网络中直接使用3 × 3卷积来实现位置编码。

传统的FFN在每个位置上都采用相同的非线性变换,而Mix-FFN则允许对不同位置应用不同的非线性变换,从而增强模型的表达能力。
具体来说,Mix-FFN使用了全局前馈神经网络(Global FFN)和局部前馈神经网络(Local FFN)两种不同的前馈神经网络结构。
全局FFN是一个具有较大感受野的前馈神经网络,能够更好地捕捉全局上下文信息。
而局部FFN是一个具有较小感受野的前馈神经网络,能够更好地捕捉局部细节信息。 通过同时使用全局FFN和局部FFN,Mix-FFN能够在处理不同位置的特征时更加灵活和准确。
全局FFN可以帮助模型捕捉到更长范围的依赖关系和语义信息,而局部FFN则可以更好地处理局部细节和细微变化。
此外,Mix-FFN在每个FFN中将一个3 × 3卷积和一个MLP混合在一起,该结构可以为Transformer提供位置信息。
并且,我们使用深度可分离卷积来减少参数数量并提高效率。
Mix-FFN可以表示为:
\begin{equation}
	{\mathbf{x}_{out} = {\text{MLP(GELU}(\text{Conv}_{\text{3} \times \text{3}}(\text{MLP}(\mathbf{x}_{in}))))+ \mathbf{x}_{in}},}
	\label{eqn:mixffn}
\end{equation}

Mix-FFN如\ref{图:mixffn} 所示。
\begin{figure}[h]
	\centering
	\includegraphics[width=0.25\textwidth]{figures/mixffn.png}
	\caption{mixffn}
	\label{图:mixffn}
\end{figure}


\textbf{(4)网络结构}
网络结构如\ref{图:efficient} 所示。
\begin{figure}[h]
	\centering
	\includegraphics[width=\textwidth]{figures/efficient.png}
	\caption{efficient}
	\label{图:efficient}
\end{figure}





\subsection{基于交叉注意力的跨模态特征融合}
\textbf{(1)基于空间和通道注意力的跨模态特征选择}

空间注意力和通道注意力可以从特定模态中压缩特征并选择特征,提高语义分割的准确性。但是,现有的空间注意力和通道注意力机制采取不可学习的方法来压缩特征,这种方法对单模态的特征选择较为充分,但是对多模态输入,就无法兼顾不同模态特征之间的差异性,进而不利于不同模态信息的利用。

针对上述问题,本文提出基于空间和通道注意力的跨模态特征选择方法。该方法通过可学习的策略对彩色信息和深度信息进行特征压缩和特征选择。

首先,对彩色信息和深度信息拼接的特征图进行均值池化和最大池化,并将两种池化信息加权求和得到特征图的全局信息向量。其次,在通道注意力部分,全局信息向量被输入到一个多层感知机用来产生表示不同通道的权重分配向量,然后将权重分配向量通过Sigmoid函数得到归一化的通道注意力权重。在空间注意力部分,全局信息向量被输入到另一个多层感知机产生表示不同通道的权重分配向量,通过与原始特征图相乘后在通道维度的相加,可以得到不同空间的空间权重分配向量,然后将空间权重分配向量通过Sigmoid函数得到归一化的空间注意力权重。最后,将原始特征图和通道注意力权重和空间注意力权重相乘得到跨模态的融合特征。

如图如\ref{图:特征选择}所示。
\begin{figure}[h]
	\centering
	\includegraphics[width=\textwidth]{figures/特征选择.png}
	\caption{基于空间和通道注意力引导的跨模态特征选择}
	\label{图:特征选择}
\end{figure}


\textbf{(2)基于交叉注意力的跨模态特征嵌入}

VIT利用transformer中的多头注意力(MultiHead Self-Attention, MHSA)对输入的模态信息进行特征提取。
但是MHSA只接受单一模态的信息,因而只能对同一模态的信息进行自相似性计算。
在跨模态的语义分割任务中,单一模态的自相似性计算无法对来自不同模态的信息进行特征提取,因此无法充分利用来自不同模态信息的互补性来提高语义分割的性能。


针对这一问题,本文提出基于交叉注意力的跨模态特征嵌入方法。
该方法借鉴自注意力机制中的自相似性计算,通过定义跨模态自相似性构造交叉注意力机制,提出一个跨模态特征嵌入模块,从而对彩色信息和深度信息进行特征融合。

\textbf{特征混合重组。}
跨模态特征嵌入模块有三个输入,分别是彩色特征,深度特征和融合特征。
首先,彩色特征和深度特征会被投影到向量子空间中,用来产生对应模态的$Key$和$Query$。
融合特征会被投影到第三个向量子空间中,用来产生融合模态对应的$Value$。
如XX所示。


然后,如果在不同的子空间中计算自相似性,那么就无法在不同的子空间同时包含彩色信息和深度信息包含的特征。
因此,为了可以从不同的特征子空间学习特征,利用混合重组方法将彩色信息和深度信息产生的$Key$拼接后打乱重组。
这样,新产生的$Key$就同时包含了来自彩色模态和深度模态的信息。将彩色信息和深度信息产生的$Query$拼接后打乱重组。这样,新产生的$Query$也同时包含了来自彩色模态和深度模态的信息。
如XX所示。

\textbf{跨模态自相似性。}
假设彩色信息和深度信息的特征被编码为 $Key$ 和 $Query$,那么对于任意一个像素 $(i_0,j_0)$,它与其他像素 $(i,j)$ 的跨模态自相似性可以被定义为:
\begin{equation}
	W(i,j)=\sum_{n=1}^{N}(Krgb_{n,i,j} \cdot Qrgb_{n,i_0,j_0})+\sum_{n=1}^{N}(Kdepth_{n,i,j} \cdot Qdepth_{n,i_0,j_0})\
\end{equation}

其中, 
$Krgb_{n,i,j}$表示像素$(i_0,j_0)$产生的$Key$在彩色模态的第$n$个特征值,
$Qrgb_{n,i,j}$表示像素$(i_0,j_0)$产生的$Query$在彩色模态的第$n$个特征值,
$Kdepth_{n,i,j}$表示像素$(i_0,j_0)$产生的$Key$在深度模态的第$n$个特征值,
$Qdepth_{n,i,j}$表示像素$(i_0,j_0)$产生的$Query$在深度模态的第$n$个特征值。

\textbf{跨模态交叉注意力。}
在计算完跨模态自相似性后,还需要将计算的结果嵌入到融合特征$Value$中。
首先,计算$K_1$和$Q_1$的点积、$K_2$和$Q_2$的点积,
并将点积结果通过Softmax函数归一化,得到特征子空间$W_1$和$W_2$。
\begin{equation}
	\begin{aligned}
		W_1=\text{Softmax}(\frac{Q_1 \cdot K_{1}^T}{\sqrt{C_1/4}})\\
		W_2=\text{Softmax}(\frac{Q_2 \cdot K_{2}^T}{\sqrt{C_1/4}})\\
	\end{aligned}
\end{equation}

然后,通过点积运算将信息嵌入到$V_1$和$V_2$中后,
将$V_1$和$V_2$在通道维度拼接可以得到最后的融合特征$Fused$。
\begin{equation}
	Fused=\text{Cat}[W_1 \cdot V_1,W_2 \cdot V_2]\\
\end{equation}

最后,将融合特征$Fused$与$Fused_1$进行残差连接,完成跨模态交叉注意力的全过程,
得到最终的融合模态$Fused_2$。
\begin{equation}
	Fused_2 = Fused + Fused_1\\
\end{equation}

\textbf{(3)融合结构}
如图如\ref{图:跨模态融合结构}所示。
\begin{figure}[h]
	\centering
	\includegraphics[width=\textwidth]{figures/跨模态融合结构.png}
	\caption{跨模态融合结构}
	\label{图:跨模态融合结构}
\end{figure}


\subsection{轻量级MLP解码器}
对于语义分割来说最重要的问题就是如何增大有效感受野。
对于Transformer encoder来说,由于self-attention有效感受野变得非常大,因此decoder不需要更多操作来提高有效感受野,也因此可以设计更加简单的decoder。

本文设计一个轻量级MLP解码器。
该解码器仅由MLP层组成,避免了其他方法中通常使用的手工设计和计算量较大的组件。
提出的全MLP解码器包括四个主要步骤。
首先,来自MiT编码器的多级特征Fi通过一个MLP层进行通道维度统一。
然后,在第二步中,特征被上采样到1/4大小,并进行拼接。
其次,采用MLP层来融合拼接后的特征F。
最后,另一个MLP层将融合后的特征输入,预测分割掩码M,分辨率为H/4 × W/4 × Ncls,其中Ncls是类别的数量。

可以将解码器表示为:
\begin{equation}
	\begin{aligned}
		&{\hat{F}_i = \text{Linear}(C_i, C)(F_i)}, \forall i\\
		&{{\hat{F}}_i = \text{Upsample}(\frac{W}{4} \times \frac{W}{4})({\hat{F}}_i)}, \forall i\\
		&{{F} = \text{Linear}(4C, C)(\text{Concat}({\hat{F}}_i})), \forall i\\
		&{{M} = \text{Linear}(C, N_{cls})({F})},
	\end{aligned}
	\label{eqn:mlp}
\end{equation}

其中,${\rm M}$是XX,$\text{Linear}(C_{in}, C_{out})(\cdot)$是XX,ZZ是XX。

如图如\ref{图:decoder}所示。
\begin{figure}[h]
	\centering
	\includegraphics[width=\textwidth]{figures/decoder.png}
	\caption{decoder}
	\label{图:decoder}
\end{figure}







\section{实验结果分析}
\subsection{数据集与评价指标}
\textbf{(1)数据集}

本张使用公开数据集NYUv2和私有的地铁排爆数据集。
NYUv2是语义分割任务中使用最广泛的基准。
NYUv2原始数据集的数据来自3个城市的464个场景,并且绝大多数场景是室内场景,共894个类别标注。
通过对原始的语义标签扩展,有13类和40类两种版本用于语义分割。
根据做语义分割任务的大多数论文设定,本文采用40类的版本。
该版本的语义标签主要包括墙壁、地板、窗户、桌子和椅子等室内物体。
数据集主要包括彩色图片、深度图片以及标注图片。
1449张精细标注的图片被进一步分为795张和654张,分别用于训练和测试。图像尺寸640×480。
地铁排爆数据集是针对地铁排爆场景的自制数据集。
基础场景是某城市的某个地铁站点,通过在地铁月台和地铁车厢内部布置管状模拟爆炸物、模拟爆炸物疑似藏匿箱体等物体,模拟真实的地铁排爆场景。
该数据集一共有XX个语义分割类别,主要包括地铁闸机、地铁月台、地铁车厢内部座椅、模拟爆炸物、模拟爆炸物疑似藏匿箱体等物体。
数据集格式依照NYUv2数据集设置,主要包括彩色图片、深度图片以及标注图片。
XX张精细标注的图片被进一步分为XX张和XX张,分别用于训练和测试。图像尺寸640×480。
两个数据集的具体对比如XX所示。

\textbf{(2)评价指标}

实验中使用的X个指标来衡量语义分割算法的性能。
第一个是参数量(),该指标反应,该指标越低越好。
第二个是FLOPs。该指标越低越好。
第三个是Miou。该指标越高越好。

\subsection{参数设置}
本章算法使用pytorch框架,使用一台服务器进行训练和测试。该服务器装配有XX型号的CPU,四张RTX4090GPU,XX版本的CUDA。
表XX显示的是本章算法在NYUv2数据集和地铁排爆数据集上的详细参数设置。
在NYUv2数据集上,GPU数量设置为XX,批大小设置为XX,训练XX个epoch。
学习策略设置为XX,初始学习率设置为XX,学习率衰减参数设置为XX,
优化方法设置为XX,损失函数设置为XX。
在地铁排爆数据集上,GPU数量设置为XX,批大小设置为XX,训练XX个epoch。
学习策略设置为XX,初始学习率设置为XX,学习率衰减参数设置为XX,
优化方法设置为XX,损失函数设置为XX。
此外,在两个数据集的训练初始阶段,都采用数据增加对原始的彩色图片和深度图片进行处理,采用XX方法来提高模型的学习能力和泛化能力,但是在测试时,使用原始的图片,不涉及任何的数据增强,也不涉及任何对图片大小进行改变的操作。


\subsection{消融实验}
本小节通过在XX数据集上进行消融实验,验证XX算法中不同模块的有效性。
因此,本章的算法
XX模块表示XX。
XX模块表示XX。
XX模块表示XX。
%这里插入小消融实验的表格。
表XX展示了XX算法消融实验的结果。

\subsection{模型性能对比}




\subsection{对比实验}


\section{本章小结}
本章通过提出XX算法,该算法通过XX解决了XX问题。
实验结果表明,XX算法的XX指标相较基准算法分别降低了XX。



\input{data/chap05}

% 致谢
\input{data/ack}                 
\cleardoublepage                 % 保证章节从奇数页开始
\phantomsection                  % 创建虚拟节以便添加目录条目时正确定位

% 文献
\addcontentsline{toc}{chapter}{参考文献} % 添加文献作为一个章节
\bibliographystyle{bstutf8}             % 定义文献样式
\bibliography{ref/refs}                 % 文献数据库

% 成果
\input{data/resume}             

% 明评
\input{data/reviewinfo}        
 
% 附录
\appendix                        
\backmatter              % 应用backmatter样式
\input{data/appendix01}

\end{document}
