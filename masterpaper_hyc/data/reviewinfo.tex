\begin{reviewinfo}
  
\begin{table}[tbh]
	\centering
	\begin{tabular}{|c|c|c|c|c|c|c|c|c|c|}
		\hline
		\hei{序号} &		\hei{评阅人} & 		\hei{职称}	& 	\hei{导师类型}	& 	\hei{\makecell{工作\\单位}}	
		& 	\hei{总分}	& 	\hei{结论}	& 	\hei{答辩建议} &		\hei{\makecell{熟悉\\程度}}& 	\hei{备注} \\

		\hline
		1 & 张三 & 教授& 博导& \makecell{XXX\\大学} &	95.8 &	达到	& \makecell{无需修改\\直接答辩}	& \makecell{有深入\\了解} &	  \\  \hline
	    2 & 李四 & 研究员& 硕导& \makecell{XXX\\大学} &	95 &	达到	& \makecell{修改后\\答辩}	& \makecell{有深入\\了解} &	\\ \hline
	    3 & 王五 & 教授&  博导& & & & & & \\ \hline
	    4 & 赵六 & 教授&  博导& & & & & & \\ \hline
	    \multirow{2}*{5} & 孙六 & 教授&  博导&  \makecell{XXX\\大学} &	59 &	尚未达到	& \makecell{修改后\\复评}	& \makecell{有深入\\了解} & \\ \cline{2-10}
	    & 孙六 & 教授&  博导& \makecell{XXX\\大学} & 80 &	尚未达到	& \makecell{无需修改\\直接答辩}	& \makecell{有深入\\了解} & \makecell{复评\\结果} \\
		\hline
	\end{tabular}
	%\caption{}
\end{table}
\wuhao{
\noindent \textbf{说明}:

1. 结论选项包括2个:“达到博士学位论文要求”、“尚未达到博士学位论文要求”。

2. 答辩建议选项包括4个:“无需修改直接答辩”、“修改后答辩”、“修改后复评”、“不予答辩”。

3. 熟悉程度选项包括3个:“有深入了解”、“比较熟悉”、“一般了解”。
\\

\color{red}{
\noindent 提醒(正式成文后删除):

1. 评阅版论文删除此页。

2. 采用双盲评阅方式的学位申请人撰写的学位论文删除此页。

3. 评阅总分无需取整。

4. 工作单位填至学校、科研院所即可。

}


}



  
\end{reviewinfo}
